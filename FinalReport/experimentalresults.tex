
Put your Experimental results
\section{Testing Frame Construction(FrameObj)}
%Rebecca
% not sure how much of this le will go over...so super brief here
An easy way of being able to integrate with the other teams seemed to be through the creation of a class, FrameObj, which would not only provide the other modules with a means of creating a frame but a list of properties that could be called for easy debugging. There are two ways that an instance of the class FrameObj could be constructed,  from the basic requirements of the frame (Frame type, Sender ID number , Receiver ID number, and Data) or from the bits of data that make up a frame.

4 sections of testing that done to frame object. 

section 1 tests the ways we can make an invalid frame







\section{Transmitting and receiving over the air}
%Renato
\subsection{Using lab2 physical layer}
\subsection{Using Team 4 physical layer}
\section{End-to-End Local Machine Testing }
%Renato

\begin{table}
	\centering
		\begin{tabular}{| c | c | }
		\hline                       
		Frame Type UE & Number Received\\
		\hline
			ACK & 2\\
			DATA & 102\\
			corrupt DATA & 2\\
			INVALID & 3466\\
			other & 0\\
		\hline
		\end{tabular}
	\caption{Table of frames received with poor transmission quality and a small hamming distance between frame types}
	\label{tab:2ACK}
\end{table}

\begin{table}
	\centering
		\begin{tabular}{| c | c | }
		\hline                       
		Frame Type & Number Received\\
		\hline
			ACK & 0\\
			DATA & 165\\
			corrupt DATA & 4\\
			INVALID & 3559\\
			other & 0\\
		\hline
		\end{tabular}
	\caption{Table of frames received with poor transmission quality and a larger hamming distance between frame types}
	\label{tab:0ACK}
\end{table}